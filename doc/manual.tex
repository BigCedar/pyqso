% PyQSO User Manual

%    Copyright (C) 2013 Christian Jacobs.

%    This file is part of PyQSO.

%    PyQSO is free software: you can redistribute it and/or modify
%    it under the terms of the GNU General Public License as published by
%    the Free Software Foundation, either version 3 of the License, or
%    (at your option) any later version.
%
%    PyQSO is distributed in the hope that it will be useful,
%    but WITHOUT ANY WARRANTY; without even the implied warranty of
%    MERCHANTABILITY or FITNESS FOR A PARTICULAR PURPOSE.  See the
%    GNU General Public License for more details.
%
%    You should have received a copy of the GNU General Public License
%    along with PyQSO.  If not, see <http://www.gnu.org/licenses/>.

\documentclass[11pt, a4paper]{report}
\usepackage[margin=1.2in]{geometry}
\usepackage{graphicx}
\usepackage{float}
\usepackage{hyperref}
\hypersetup{
  colorlinks=false,
  pdfborder={0 0 0},
  }

\setlength{\parskip}{0.25cm}

\begin{document}

\begin{titlepage}
\begin{center}
\vspace*{5cm}
\huge{PyQSO User Manual}\\\vspace*{5cm}
\LARGE{Version 0.1a.dev}
\end{center}
\end{titlepage}

\tableofcontents

\chapter{Introduction}
PyQSO is a contact logging tool for amateur radio operators. It is currently in the early stages of development.

As the name suggests, PyQSO is written primarily in the Python programming language. The graphical user interface (GUI) has been built using the GTK+ library through the PyGTK bindings. PyQSO also uses an SQLite embedded database to manage all the contacts an amateur radio operator makes.

Many amateur radio operators choose to store all the contacts they ever make in a single \textit{logbook}, whereas others keep a separate logbook for each year, for example. Each logbook may be divided up to form multiple distinct \textit{logs}, perhaps one for casual repeater contacts and another for DX'ing. Finally, each log can contain multiple \textit{records}. PyQSO is based around this three-tier model, going from logbooks at the top to individual records at the bottom. From an implementation point-of-view, a database is analogous to a logbook, a table in the database is analogous to a log in the logbook, and the records in each table are analogous to the records in each log.

\section{Licensing}
PyQSO is free software, released under the GNU General Public License. Please see the file called COPYING for more information.

\section{Installation and running}
Assuming that your current working directory is PyQSO's base directory (the directory that the Makefile is in), you can install PyQSO via the terminal with the following command:

  \texttt{make install}

\noindent Note: you may need to use sudo for this. Once installed, the following command will run PyQSO:

  \texttt{pyqso}

\noindent Alternatively, PyQSO can be run (without installing) with:

  \texttt{python bin/pyqso}

\noindent from PyQSO's base directory.



\chapter{Getting started}

\section{Command-line options}
There are several options available when executing PyQSO from the command-line.

\subsection{Load logbook file}
In addition to being able to load a logbook through the GUI, users can also specify a logbook file to load at the command line with the \texttt{-l} or \texttt{--logbook} option. For example, to load a logbook file called \texttt{mylogbook.db}, use the following command:

  \texttt{pyqso --logbook /path/to/mylogbook.db}

\subsection{Debugging mode}
Running PyQSO with the \texttt{-d} or \texttt{--debug} flag enables the debugging mode:

  \texttt{pyqso --debug}

\noindent All debugging-related messages are written to a file called pyqso.debug, located in the current working directory.


\section{Creating a new logbook}
Logbooks are SQL databases, and as such they must be accessed with a database connection.

To create a connection, click \texttt{Connect to Logbook...} in the \texttt{Logbook} menu, and either:
\begin{itemize}
  \item Find and select an existing logbook database file (with the \texttt{.db} file extension), and click \texttt{Open} to create the database connection; or
  \item Create a new database by entering a (non-existing) file name and clicking \texttt{Open}. The logbook database file (and a connection to it) will then be created automatically.
\end{itemize}
Once the database connection has been established, the database file name will appear in the status bar. All logs in the logbook will be opened automatically.

\section{Log management}
\subsection{New log}
To create a new log, click \texttt{New Log} in the \texttt{Logbook} menu and enter the desired name of the log (e.g. repeater\_contacts, dx, mobile\_log). This name must be unique. Alternatively, use the shortcut key combination \texttt{Ctrl + N}.

\subsection{Renaming a log}
\subsection{Deleting a log}

\subsection{Importing and exporting a log}
While PyQSO stores logbooks in SQL format, it is possible to export individual logs in the well-known ADIF format. View the log you wish to export, and click \texttt{Export Log} in the \texttt{Logbook} menu.

Similarly, records can be imported from an ADIF file. Upon importing, users can choose to store the records in a new log, or append them to an existing log in the logbook. To import, click \texttt{Import Log} in the \texttt{Logbook} menu.

Note that all data must conform to the ADIF standard, otherwise it will be ignored.

\subsection{Printing a log}
Due to restrictions on the page width, only a selection of field names will be printed: callsign, date, time, frequency, and mode.

\subsection{Filtering by callsign}
Entering an expression such as \texttt{xyz} into the ``Filter by callsign'' box will instantly filter out any callsigns beginning with that expression.

\subsection{Sorting by field}
To sort a log by a particular field name, left-click the column header that contains that field name. By default, it is the \texttt{Index} field that is sorted in ascending order.

\section{Record management}

\subsection{New record (QSO)}
A new QSO can be added by either:
\begin{itemize}
  \item Clicking the `+' button in the toolbar.
  \item Pressing \texttt{Ctrl + R}.
  \item Clicking \texttt{Add Record...} in the \texttt{Records} menu.
\end{itemize}
A dialog window will appear where you can enter the details of the QSO. Note that the current date and time are filled in automatically.

\subsubsection{Callsign lookup}
PyQSO can also resolve callsign-related information (e.g. the operator's name, address, and ITU Zone) by querying the qrz.com database.

Note that users must first supply their qrz.com account information in the preferences dialog window.


\chapter{Toolbox}

\section{DX cluster}
A DX cluster is essentially a server through which radio operators can report and receive updates about QSOs that are in progress across the bands.

PyQSO is able to connect to a DX cluster that operates using the Telnet protocol to provide a text-based alert service. As a result of the many different Telnet-based software products that DX clusters run, PyQSO currently outputs the raw data received from the DX cluster rather than trying to parse it in some way.

\section{Grey line}
The grey line tool can be used to check which parts of the world are in darkness. The grey line window is updated every 30 minutes.

\section{Awards}

\chapter{Preferences}
PyQSO user preferences are stored in a configuration file located at \texttt{\textasciitilde/.pyqso.cfg}, where \texttt{\textasciitilde} denotes the user's home directory.

\section{View}
Not all the available fields have to be displayed in the logbook. Users can choose to hide a subset of them by unchecking them in the View tab. 

PyQSO must be restarted in order for any changes to take effect.

\section{Hamlib support}
PyQSO features rudimentary support for the Hamlib library. The name and path of the radio device connected to the user's computer can be specified in the Hamlib tab of the preferences dialog. Upon adding a new record to the log, PyQSO will use Hamlib to retrieve the current frequency that the radio device is set to and automatically fill in the Frequency field. 

\bibliographystyle{plainnat}
\end{document}
