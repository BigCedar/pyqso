% PyQSO User Manual

%    Copyright (C) 2013 Christian Jacobs.

%    This file is part of PyQSO.

%    PyQSO is free software: you can redistribute it and/or modify
%    it under the terms of the GNU General Public License as published by
%    the Free Software Foundation, either version 3 of the License, or
%    (at your option) any later version.
%
%    PyQSO is distributed in the hope that it will be useful,
%    but WITHOUT ANY WARRANTY; without even the implied warranty of
%    MERCHANTABILITY or FITNESS FOR A PARTICULAR PURPOSE.  See the
%    GNU General Public License for more details.
%
%    You should have received a copy of the GNU General Public License
%    along with PyQSO.  If not, see <http://www.gnu.org/licenses/>.

\documentclass[11pt, a4paper]{report}
\usepackage[margin=1.2in]{geometry}
\usepackage{graphicx}
\usepackage{float}

\setlength{\parskip}{0.25cm}

\begin{document}

\begin{titlepage}
\begin{center}
\vspace*{5cm}
\huge{PyQSO User Manual}\\\vspace*{5cm}
\LARGE{Version 0.1a.dev}
\end{center}
\end{titlepage}

\tableofcontents

\chapter{Introduction}
PyQSO is a contact logging tool for amateur radio operators. It is currently in the early stages of development.

As the name suggests, PyQSO is written in Python. The graphical user interface (GUI) has been built using the GTK+ library through the PyGTK bindings. PyQSO also uses an SQLite embedded database to manage all the contacts an amateur radio operator makes.

Many amateur radio operators choose to store all the contacts they ever make in a single \textit{logbook}, whereas others keep a separate logbook for each year, for example. Each logbook may be divided up to form multiple distinct \textit{logs}, perhaps one for casual repeater contacts and another for DX'ing. Finally, each log can contain multiple \textit{records}. PyQSO is based around this three-tier model, going from logbooks at the top to individual records at the bottom. From an implementation point-of-view, a database is analogous to a logbook, a table in the database is analogous to a log in the logbook, and the records in each table are analogous to the records in each log.

\section{Licensing}
PyQSO is free software, released under the GNU General Public License. Please see the file called COPYING for more information.

\section{Installation and running}
Assuming that your current working directory is PyQSO's base directory (the directory that the Makefile is in), you can install PyQSO via the terminal with the following command:

  \texttt{make install}

\noindent Note: you may need to use sudo for this. Once installed, the following command will run PyQSO:

  \texttt{pyqso}

\noindent Alternatively, PyQSO can be run (without installing) with:

  \texttt{python bin/pyqso}

\noindent from PyQSO's base directory.



\chapter{Getting started}

\section{Command-line options}
There are several options available when executing PyQSO from the command-line.

\subsection{Load logbook file}
In addition to being able to load a logbook through the GUI, users can also specify a logbook file to load at the command line with the \texttt{-l} or \texttt{--logbook} option. For example, to load a logbook file called \texttt{mylogbook.db}, use the following command:

  \texttt{pyqso --logbook /path/to/mylogbook.db}

\subsection{Debugging mode}
Running PyQSO with the \texttt{-d} or \texttt{--debug} flag enables the debugging mode:

  \texttt{pyqso --debug}

\noindent All debugging-related messages are written to a file called pyqso.debug, located in the current working directory. If you encounter a bug in PyQSO, feel free to submit a bug report to the developers. If it is possible to replicate the bug, please re-run PyQSO with the debugging mode enabled and submit the pyqso.debug file along with your bug report.

\section{Adding a new contact}


\chapter{DX cluster}



\bibliographystyle{plainnat}
\end{document}
